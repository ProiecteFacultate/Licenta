\begin{abstractpage}

\begin{abstract}{romanian}

Sistemele de fișiere reprezintă o componentă esențială în cadrul sistemelor de calcul, fără de care acestea nu ar putea funcționa. Ele constituie cadrul fundamental de stocare, organizare dar și gestionare a datelor utilizate de către un computer. Acestea reprezintă structurile de bază care le permit utilizatorilor scrierea, accesarea, modificarea și ștergerea, într-un mod eficient, a informațiilor aflate într-un mediu de stocare. Astfel, importanța sistemelor de fișiere nu se limitează doar la mediul informatic, ci are un impact semnificativ și în cadrul utilizării cotidiene a calculatoarelor.

În acest scop, în cadrul acestei lucrări, ne vom concentra pe explorarea detaliată a mai multor sisteme de fișiere, cum ar fi FAT32, Ext2 și Veritas, dar și a unui sistem modern de fișiere distribuit. Ne vom focusa pe aspectele distinctive ale fiecăruia dintre cele 3 sisteme de fișiere, evidențiind funcționalitățile principale ale acestora, cum ar fi crearea, modificarea, citirea și ștergerea datelor. Scopul este acela de a înțelege modul în care datele sunt organizate în cadrul unui mediu de stocare, și în special modul în care acestea sunt interpretate sub forma unor foldere și fișiere, astfel încât utilizarea lor să poată fi posibilă.
\end{abstract}

\begin{abstract}{english}
File systems are an essential component within computing systems, without which they could not function. They constitute the fundamental framework for storing, organizing and managing the data used by a computer. These are the basic structures that allow users to efficiently write, access, modify, and delete information on a storage medium. Thus, the importance of file systems is not only limited to the computer environment, but also has a significant impact in the daily use of computers.

To this end, in this paper, we will focus on the detailed exploration of several file systems such as FAT32, Ext2 and Veritas, but also a modern distributed file system. We will focus on the distinctive aspects of each of the 3 file systems, highlighting their main functionalities such as creating, modifying, reading and deleting data. The goal is to understand how data is organized within a storage medium, and in particular how it is interpreted as folders and files so that its use can be made possible.
\end{abstract}

\end{abstractpage}